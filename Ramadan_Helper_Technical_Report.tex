\documentclass[12pt,a4paper]{article}
\usepackage[utf8]{inputenc}
\usepackage[T1]{fontenc}
\usepackage{geometry}
\usepackage{graphicx}
\usepackage{hyperref}
\usepackage{listings}
\usepackage{xcolor}
\usepackage{booktabs}
\usepackage{longtable}
\usepackage{fancyhdr}
\usepackage{titlesec}
\usepackage{enumitem}
\usepackage{tcolorbox}
\usepackage{float}
\usepackage{parskip}
\usepackage{setspace}

% Set line spacing
\singlespacing

% Page geometry
\geometry{margin=0.85in}

% Colors
\definecolor{codegreen}{rgb}{0,0.6,0}
\definecolor{codegray}{rgb}{0.5,0.5,0.5}
\definecolor{codepurple}{rgb}{0.58,0,0.82}
\definecolor{backcolour}{rgb}{0.95,0.95,0.92}
\definecolor{darkgreen}{rgb}{0,0.5,0}

% Code listing style
\lstdefinestyle{mystyle}{
    backgroundcolor=\color{backcolour},
    commentstyle=\color{codegreen},
    keywordstyle=\color{magenta},
    numberstyle=\tiny\color{codegray},
    stringstyle=\color{codepurple},
    basicstyle=\ttfamily\footnotesize,
    breakatwhitespace=false,
    breaklines=true,
    captionpos=b,
    keepspaces=true,
    numbers=left,
    numbersep=5pt,
    showspaces=false,
    showstringspaces=false,
    showtabs=false,
    tabsize=2,
    frame=single
}
\lstset{style=mystyle}

% Header and footer
\pagestyle{fancy}
\fancyhf{}
\fancyhead[L]{Ramadan Helper Application}
\fancyhead[R]{Technical Report}
\fancyfoot[C]{\thepage}

% Title formatting
\titleformat{\section}{\Large\bfseries\color{darkgreen}}{\thesection}{1em}{}
\titleformat{\subsection}{\large\bfseries\color{darkgreen!80}}{\thesubsection}{1em}{}
\titleformat{\subsubsection}{\normalsize\bfseries\color{darkgreen!60}}{\thesubsubsection}{1em}{}

% Hyperlink setup
\hypersetup{
    colorlinks=true,
    linkcolor=darkgreen,
    filecolor=magenta,
    urlcolor=blue,
}

\begin{document}

% Title Page
\begin{titlepage}
    \centering
    
    % University/Institution Header (optional - customize as needed)
    \vspace*{0.5cm}
    {\large\textsc{Web Services Development Project}\par}
    \vspace{0.3cm}
    {\large Academic Year 2025-2026\par}
    
    \vspace{1.5cm}
    
    % Decorative line
    \rule{\textwidth}{1.5pt}
    \vspace{0.5cm}
    
    % Title
    {\Huge\bfseries Ramadan Helper Application\par}
    \vspace{0.3cm}
    {\LARGE RESTful API Technical Report\par}
    
    \vspace{0.5cm}
    \rule{\textwidth}{1.5pt}
    
    \vspace{2cm}
    
    % Author and Supervisor
    \begin{minipage}[t]{0.45\textwidth}
        \begin{flushleft}
            \large\textbf{Author:}\\
            \large Chiheb Bahri
        \end{flushleft}
    \end{minipage}
    \hfill
    \begin{minipage}[t]{0.45\textwidth}
        \begin{flushright}
            \large\textbf{Supervisor:}\\
            \large Montassar Ben Messaoud
        \end{flushright}
    \end{minipage}
    
    \vspace{3cm}
    
    % Project Description
    \begin{tcolorbox}[colback=green!5!white, colframe=darkgreen, title=Project Overview]
        A comprehensive web application providing Islamic guidance and community features for the holy month of Ramadan. Built with FastAPI backend and modern frontend technologies, featuring AI-powered assistance, event management, and secure user authentication.
    \end{tcolorbox}
    
    \vfill
    
    % Footer information
    {\large Version 1.0 - Final Release\par}
    \vspace{0.3cm}
    {\large January 2026\par}
    \vspace{0.5cm}
    {\small\textbf{Repository:} \url{https://github.com/cheehub213/ramadan-webservice-project-}\par}
    
\end{titlepage}

% Table of Contents
\tableofcontents
\newpage

% Introduction
\section{Introduction}

\subsection{Context and Motivation}
The holy month of Ramadan is a significant period for Muslims worldwide. With increasing digitalization, there is a growing need for accessible digital platforms to assist Muslims during Ramadan. This project develops a comprehensive web application combining modern web technologies with Islamic knowledge resources.

\subsection{Project Objectives}
\begin{enumerate}[noitemsep]
    \item \textbf{AI-Powered Islamic Guidance:} Generate contextually appropriate duas and answer Islamic questions.
    \item \textbf{Community Event Management:} Platform for discovering and sharing community events.
    \item \textbf{Secure User Authentication:} Robust authentication with email verification.
    \item \textbf{RESTful API:} Well-documented API following industry best practices.
\end{enumerate}

\subsection{Scope}
This report covers: system architecture, API specifications, database schema, security implementation, external services integration, and deployment procedures.

% Executive Summary
\section{Executive Summary}

The Ramadan Helper Application is a full-stack web platform providing AI-powered Islamic guidance, community event management, and secure user authentication. Built with FastAPI backend and responsive SPA frontend.

\subsection{Key Features}
\begin{itemize}[noitemsep]
    \item \textbf{AI-Powered Dua Generation}: Personalized supplications using Groq AI
    \item \textbf{Quran Search}: Semantic search across Quran verses
    \item \textbf{Community Events}: Create and discover local Ramadan events
    \item \textbf{User Authentication}: JWT-based auth with email verification
    \item \textbf{Admin Dashboard}: Administrative control panel
\end{itemize}

\subsection{Technology Stack}
\begin{table}[h]
\centering
\small
\begin{tabular}{@{}ll@{}}
\toprule
\textbf{Component} & \textbf{Technology} \\
\midrule
Backend & FastAPI 0.104.1 (Python 3.12) \\
Database & SQLite + SQLAlchemy ORM \\
Authentication & JWT + bcrypt \\
AI Service & Groq API \\
Email & Gmail SMTP \\
Frontend & HTML5, CSS3, JavaScript \\
\bottomrule
\end{tabular}
\end{table}

% System Architecture
\section{System Architecture}

\subsection{Overview}
The application follows a three-tier architecture: presentation layer (frontend SPA), business logic layer (FastAPI backend), and data access layer (SQLAlchemy ORM with SQLite).

\begin{verbatim}
+------------+     +------------+     +----------+
|  Frontend  |<--->|  Backend   |<--->| Database |
|  (8080)    |     |  (8000)    |     | (SQLite) |
+------------+     +-----+------+     +----------+
                        |
              +---------+---------+
              | External Services |
              | Groq AI | Gmail   |
              +-------------------+
\end{verbatim}

\subsection{Backend Structure}
\begin{verbatim}
backend/
|-- main.py, config.py, database.py
|-- models.py, models_extended.py
|-- routes/ (auth, events, admin, chat, dua)
|-- schemas/ (auth, events, chat, dua)
|-- services/ (email_service)
\end{verbatim}

% API Documentation
\section{API Documentation}

Base URL: \texttt{http://localhost:8000/api}. Interactive docs available at \texttt{/docs}.

\subsection{Authentication Endpoints}

\begin{table}[h]
\centering\small
\begin{tabular}{@{}llll@{}}
\toprule
\textbf{Endpoint} & \textbf{Method} & \textbf{Auth} & \textbf{Description} \\
\midrule
/auth/signup & POST & No & Register new user \\
/auth/login & POST & No & Login and get JWT token \\
/auth/verify-email/\{token\} & GET & No & Verify email address \\
\bottomrule
\end{tabular}
\end{table}

\textbf{Signup Request:} \texttt{\{"email": "...", "password": "...", "name": "..."\}}

\textbf{Login Response:} \texttt{\{"access\_token": "...", "token\_type": "bearer", "user": \{...\}\}}

\subsection{Events Endpoints}

\begin{table}[h]
\centering\small
\begin{tabular}{@{}llll@{}}
\toprule
\textbf{Endpoint} & \textbf{Method} & \textbf{Auth} & \textbf{Description} \\
\midrule
/events/ & GET & No & List all events \\
/events/ & POST & Yes & Create new event \\
/events/\{id\} & GET & No & Get event details \\
/events/\{id\} & DELETE & Yes & Delete event \\
\bottomrule
\end{tabular}
\end{table}

\subsection{AI Endpoints}

\begin{table}[h]
\centering\small
\begin{tabular}{@{}llll@{}}
\toprule
\textbf{Endpoint} & \textbf{Method} & \textbf{Auth} & \textbf{Description} \\
\midrule
/dua/generate & POST & No & Generate dua for situation \\
/chat/ & POST & No & AI chat for Islamic Q\&A \\
\bottomrule
\end{tabular}
\end{table}

% Database Schema
\section{Database Schema}

SQLite database with SQLAlchemy ORM. Two main tables with foreign key relationship.

\subsection{Entity Relationship}
\begin{verbatim}
User (1) -----> (*) Event
  |                  |
  +--organizer_id----+
\end{verbatim}

\subsection{Tables}

\begin{table}[h]
\centering\small
\begin{tabular}{@{}lll@{}}
\toprule
\textbf{User Table} & \textbf{Type} & \textbf{Description} \\
\midrule
id & Integer (PK) & Primary key \\
email & String & Unique, indexed \\
hashed\_password & String & bcrypt hash \\
name & String & Display name \\
is\_active & Boolean & Account status \\
is\_admin & Boolean & Admin privileges \\
is\_verified & Boolean & Email verified \\
verification\_token & String & Email token \\
created\_at & DateTime & Creation timestamp \\
\bottomrule
\end{tabular}
\end{table}

\begin{table}[h]
\centering\small
\begin{tabular}{@{}lll@{}}
\toprule
\textbf{Event Table} & \textbf{Type} & \textbf{Description} \\
\midrule
id & Integer (PK) & Primary key \\
title & String & Event title \\
description & Text & Event details \\
date, time & String & Event schedule \\
location & String & Event location \\
organizer\_id & Integer (FK) & User reference \\
organizer\_name & String & Cached name \\
is\_featured & Boolean & Featured flag \\
created\_at & DateTime & Creation timestamp \\
\bottomrule
\end{tabular}
\end{table}

% Security Implementation
\section{Security Implementation}

\subsection{Authentication Flow}
\begin{enumerate}[noitemsep]
    \item User registers with email/password
    \item Password hashed with bcrypt (salted)
    \item Verification email sent via SMTP
    \item User verifies email via link
    \item Login returns JWT token (30-min expiry)
    \item Token sent in Authorization header
\end{enumerate}

\subsection{Password Hashing}
\begin{lstlisting}[language=Python]
import bcrypt

def get_password_hash(password: str) -> str:
    return bcrypt.hashpw(password.encode(), bcrypt.gensalt()).decode()

def verify_password(plain: str, hashed: str) -> bool:
    return bcrypt.checkpw(plain.encode(), hashed.encode())
\end{lstlisting}

\subsection{JWT Token Generation}
\begin{lstlisting}[language=Python]
from jose import jwt
from datetime import datetime, timedelta

def create_access_token(data: dict) -> str:
    expire = datetime.utcnow() + timedelta(minutes=30)
    return jwt.encode({**data, "exp": expire}, SECRET_KEY, "HS256")
\end{lstlisting}

\subsection{Security Best Practices}
\begin{itemize}[noitemsep]
    \item CORS middleware configuration
    \item Environment variables for secrets
    \item Password hashing (never stored plain)
    \item Token expiration
    \item Email verification required
    \item Role-based access control
\end{itemize}

% External Services Integration
\section{External Services Integration}

The Ramadan Helper Application integrates with external services to provide enhanced functionality beyond basic CRUD operations. This section describes the integration architecture and implementation details for each external service.

\subsection{Groq AI API}

The application uses Groq AI for intelligent features including dua generation and Islamic Q\&A. Groq provides fast inference for large language models, making it suitable for real-time conversational features.

\textbf{Integration Architecture:}
\begin{itemize}
    \item API calls are made from the backend to prevent exposure of API keys
    \item Responses are processed and formatted before being sent to the frontend
    \item Error handling ensures graceful degradation if the AI service is unavailable
\end{itemize}

\begin{lstlisting}[language=Python, caption=Groq AI Service Integration]
from groq import Groq

client = Groq(api_key=os.getenv("GROQ_API_KEY"))

def generate_dua(situation: str, language: str = "english"):
    response = client.chat.completions.create(
        model="llama-3.1-70b-versatile",
        messages=[
            {
                "role": "system",
                "content": "You are an Islamic scholar..."
            },
            {
                "role": "user",
                "content": f"Generate a dua for: {situation}"
            }
        ],
        temperature=0.7,
        max_tokens=1000
    )
    return response.choices[0].message.content
\end{lstlisting}

\subsection{Gmail SMTP Service}

\begin{lstlisting}[language=Python, caption=Email Service Configuration]
import smtplib
from email.mime.text import MIMEText

SMTP_SERVER = "smtp.gmail.com"
SMTP_PORT = 587
SMTP_USER = os.getenv("SMTP_USER")
SMTP_PASSWORD = os.getenv("SMTP_PASSWORD")

def send_verification_email(to_email: str, token: str):
    verification_url = f"{BASE_URL}/verify-email/{token}"
    
    msg = MIMEText(f"""
    Please verify your email by clicking:
    {verification_url}
    """)
    msg['Subject'] = 'Verify Your Ramadan Helper Account'
    msg['From'] = SMTP_USER
    msg['To'] = to_email
    
    with smtplib.SMTP(SMTP_SERVER, SMTP_PORT) as server:
        server.starttls()
        server.login(SMTP_USER, SMTP_PASSWORD)
        server.send_message(msg)
\end{lstlisting}

\newpage

% Frontend Architecture
\section{Frontend Architecture}

The frontend of the Ramadan Helper Application is designed as a single-page application (SPA) that provides a seamless user experience without full page reloads. This section describes the frontend architecture and key implementation decisions.

\subsection{Single Page Application Design}

The frontend is implemented as a single HTML file with embedded CSS and JavaScript, providing a responsive and interactive user experience. This approach was chosen for its simplicity and ease of deployment, while still providing modern SPA functionality.

\textbf{Advantages of this approach:}
\begin{itemize}
    \item No build process required for development
    \item Easy deployment to any static file server
    \item Minimal dependencies and fast loading times
    \item Full control over the application lifecycle
\end{itemize}

\subsection{Key Components}

\begin{itemize}
    \item \textbf{Navigation}: Tab-based navigation between sections
    \item \textbf{Authentication Modal}: Login/Signup forms
    \item \textbf{Events Display}: Featured and regular events listing
    \item \textbf{Event Creation Form}: For authenticated users
    \item \textbf{AI Chat Interface}: Interactive chat with AI assistant
    \item \textbf{Admin Panel}: User and event management (admin only)
\end{itemize}

\subsection{API Communication}

\begin{lstlisting}[language=JavaScript, caption=Frontend API Helper Function]
async function apiCall(endpoint, method = 'GET', body = null) {
    const headers = {
        'Content-Type': 'application/json'
    };
    
    const token = localStorage.getItem('token');
    if (token) {
        headers['Authorization'] = `Bearer ${token}`;
    }
    
    const options = { method, headers };
    if (body) {
        options.body = JSON.stringify(body);
    }
    
    const response = await fetch(
        `http://localhost:8000/api${endpoint}`, 
        options
    );
    
    if (!response.ok) {
        throw new Error(`HTTP error! status: ${response.status}`);
    }
    
    return response.json();
}
\end{lstlisting}

\subsection{State Management}

\begin{lstlisting}[language=JavaScript, caption=Local Storage State Management]
// Store authentication state
function saveAuthState(token, user) {
    localStorage.setItem('token', token);
    localStorage.setItem('user', JSON.stringify(user));
}

// Retrieve authentication state
function getAuthState() {
    const token = localStorage.getItem('token');
    const user = JSON.parse(localStorage.getItem('user') || 'null');
    return { token, user };
}

// Clear authentication state
function clearAuthState() {
    localStorage.removeItem('token');
    localStorage.removeItem('user');
}
\end{lstlisting}

\newpage

% Testing
\section{Testing}

Testing is essential for ensuring the reliability and correctness of the application. This section describes the testing strategy and available test suites for the Ramadan Helper Application.

The project includes multiple test files targeting different aspects of the application functionality. Tests are written using Python's pytest framework and can be executed both individually and as a complete suite.

\subsection{Test Files Overview}

\begin{table}[h]
\centering
\begin{tabular}{@{}ll@{}}
\toprule
\textbf{Test File} & \textbf{Purpose} \\
\midrule
test\_simple\_app.py & Basic endpoint connectivity \\
test\_dua\_only.py & Dua generation functionality \\
test\_chat\_dua.py & AI chat integration \\
test\_semantic\_search.py & Quran search feature \\
test\_imam\_registration.py & Imam registration flow \\
test\_new\_endpoints.py & New API endpoints \\
\bottomrule
\end{tabular}
\caption{Test Suite Overview}
\end{table}

\subsection{Running Tests}

\begin{lstlisting}[language=bash, caption=Test Execution Commands]
# Run all tests
python -m pytest

# Run specific test file
python -m pytest test_simple_app.py -v

# Run with coverage report
python -m pytest --cov=backend --cov-report=html
\end{lstlisting}

\newpage

% Deployment
\section{Deployment}

This section provides comprehensive instructions for deploying the Ramadan Helper Application in various environments, from local development to production deployment.

\subsection{Local Development Setup}

Follow these steps to set up the development environment on your local machine:

\begin{lstlisting}[language=bash, caption=Development Environment Setup]
# Clone repository
git clone https://github.com/cheehub213/ramadan-webservice-project-.git
cd ramadan-webservice-project-

# Create virtual environment
python -m venv .venv
.venv\Scripts\Activate.ps1  # Windows
source .venv/bin/activate    # Linux/Mac

# Install dependencies
pip install -r backend/requirements.txt

# Configure environment
# Create backend/.env with:
# GROQ_API_KEY=your_api_key
# SMTP_USER=your_email@gmail.com
# SMTP_PASSWORD=your_app_password
# SECRET_KEY=your_secret_key

# Start backend server
cd backend
python main.py

# Start frontend (new terminal)
cd app/schemas/frontend\ webservice\ site
python -m http.server 8080
\end{lstlisting}

\subsection{Docker Deployment}

\begin{lstlisting}[language=bash, caption=Docker Compose Configuration]
# docker-compose.yml
version: '3.8'
services:
  backend:
    build: .
    ports:
      - "8000:8000"
    environment:
      - GROQ_API_KEY=${GROQ_API_KEY}
      - SMTP_USER=${SMTP_USER}
      - SMTP_PASSWORD=${SMTP_PASSWORD}
    volumes:
      - ./data:/app/data
\end{lstlisting}

\subsection{Production Considerations}

\begin{itemize}
    \item Replace SQLite with PostgreSQL for production
    \item Configure HTTPS with SSL certificates
    \item Set up reverse proxy (nginx/Apache)
    \item Implement rate limiting
    \item Configure proper CORS origins
    \item Use production WSGI server (Gunicorn/Uvicorn workers)
\end{itemize}

\newpage

% Troubleshooting
\section{Troubleshooting}

During development and deployment, various issues may arise. This section documents common problems encountered and their solutions, serving as a reference for developers and system administrators.

\subsection{Common Issues and Solutions}

\subsubsection{bcrypt Compatibility Error}
\textbf{Issue:} \texttt{AttributeError: module 'bcrypt' has no attribute '\_\_about\_\_'}

\textbf{Solution:} Use bcrypt directly instead of passlib:
\begin{lstlisting}[language=Python]
import bcrypt
# Instead of passlib.context.CryptContext
\end{lstlisting}

\subsubsection{Events Not Displaying}
\textbf{Issue:} Posted events don't appear in the events list

\textbf{Solution:} Ensure JavaScript element IDs match HTML:
\begin{lstlisting}[language=JavaScript]
// Correct element ID
document.getElementById('eventsList')
// Not 'eventsContainer'
\end{lstlisting}

\subsubsection{CORS Errors}
\textbf{Issue:} Cross-origin request blocked

\textbf{Solution:} Configure CORS in FastAPI:
\begin{lstlisting}[language=Python]
from fastapi.middleware.cors import CORSMiddleware

app.add_middleware(
    CORSMiddleware,
    allow_origins=["http://localhost:8080"],
    allow_credentials=True,
    allow_methods=["*"],
    allow_headers=["*"],
)
\end{lstlisting}

\newpage

% Conclusion
\section{Conclusion}

\subsection{Project Achievements}

The Ramadan Helper Application successfully achieves its primary objectives of providing a comprehensive digital platform for Muslims during Ramadan. The key accomplishments of this project include:

\begin{itemize}
    \item \textbf{Functional RESTful API:} A well-structured API with proper authentication, authorization, and error handling that follows industry best practices.
    \item \textbf{AI Integration:} Successful integration with Groq AI for intelligent dua generation and Islamic Q\&A functionality.
    \item \textbf{Secure Authentication:} Implementation of JWT-based authentication with bcrypt password hashing and email verification.
    \item \textbf{Community Features:} Event management system allowing users to create, share, and discover community events.
    \item \textbf{Responsive Frontend:} A user-friendly single-page application that works across different devices.
\end{itemize}

\subsection{Lessons Learned}

During the development of this project, several valuable lessons were learned:

\begin{enumerate}
    \item \textbf{Library Compatibility:} The importance of checking library compatibility with Python versions, as demonstrated by the bcrypt/passlib issue with Python 3.12.
    \item \textbf{API Design:} The value of consistent API design patterns for maintainability and ease of frontend integration.
    \item \textbf{Error Handling:} The necessity of comprehensive error handling and meaningful error messages for debugging.
\end{enumerate}

\subsection{Future Improvements}

Potential enhancements for future versions include:

\begin{itemize}
    \item Migration to PostgreSQL for production deployment
    \item Implementation of real-time notifications using WebSockets
    \item Mobile application development using React Native
    \item Integration with prayer time APIs for location-based features
    \item Multi-language support for international users
\end{itemize}

\newpage

% Appendix
\section{Appendix}

\subsection{Environment Variables Reference}

\begin{table}[h]
\centering
\begin{tabular}{@{}lll@{}}
\toprule
\textbf{Variable} & \textbf{Description} & \textbf{Required} \\
\midrule
GROQ\_API\_KEY & Groq AI API key & Yes \\
SMTP\_USER & Gmail address & Yes \\
SMTP\_PASSWORD & Gmail app password & Yes \\
SECRET\_KEY & JWT signing key & Yes \\
DEBUG & Enable debug mode & No \\
DATABASE\_URL & Database connection & No \\
\bottomrule
\end{tabular}
\caption{Environment Variables}
\end{table}

\subsection{API Status Codes}

\begin{table}[h]
\centering
\begin{tabular}{@{}ll@{}}
\toprule
\textbf{Code} & \textbf{Description} \\
\midrule
200 & Success \\
201 & Created \\
400 & Bad Request \\
401 & Unauthorized \\
403 & Forbidden \\
404 & Not Found \\
422 & Validation Error \\
500 & Internal Server Error \\
\bottomrule
\end{tabular}
\caption{HTTP Status Codes}
\end{table}

\subsection{Version History}

\begin{table}[h]
\centering
\begin{tabular}{@{}lll@{}}
\toprule
\textbf{Version} & \textbf{Date} & \textbf{Changes} \\
\midrule
v0 & Jan 2026 & Initial release with all features \\
v1.0 & Jan 2026 & Bug fixes: auth, events display \\
\bottomrule
\end{tabular}
\caption{Version History}
\end{table}

\vfill

\begin{center}
\rule{0.5\textwidth}{0.5pt}

\vspace{0.5cm}
\textit{End of Technical Documentation}

\vspace{0.3cm}
{\small Ramadan Helper Application v1.0}

{\small Developed by Chiheb Bahri}

{\small Under the supervision of Montassar Ben Messaoud}
\end{center}

\end{document}
